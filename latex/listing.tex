\documentclass[12pt, a4paper]{scrartcl}

\usepackage[utf8]{inputenc}
\usepackage[ngerman]{babel}
\usepackage[T1]{fontenc}
\usepackage{listings}
\usepackage{xcolor}
\usepackage{hyperref}

% Setzen einiger Optionen für listings
\lstset{
  backgroundcolor=\color{lightgray!20!white},
  basicstyle=\scriptsize\ttfamily,
  keywordstyle=\bfseries\ttfamily\color{orange!40!black},
  stringstyle=\color{blue}\ttfamily,
  commentstyle=\color{green!40!black}\ttfamily,
  emph={square}, 
  emphstyle=\color{blue}\texttt,
  emph={[2]root,base},
  emphstyle={[2]\color{yac}\texttt},
  showstringspaces=false,
  flexiblecolumns=false,
  tabsize=2,
  numbers=left,
  numberstyle=\tiny,
  numberblanklines=false,
  stepnumber=1,
  numbersep=10pt,
  xleftmargin=15pt,
  breaklines=true,
  prebreak={\mbox{$\hookleftarrow$}},  %% Vor Zeilenumbruch Zeichen setzen
  breakautoindent=true         %% umbrochene Zeilen einrücken 
}


\title{Beispiel für listing}
\author{Robert Lehmann}

\begin{document}
\maketitle
\tableofcontents
\lstlistoflistings

\section{\texttt{lstinputlisting}}
%=====================================
Der Befehl \texttt{lstinputlisting} wird wie in Listing \ref{lst:use_input} 
verwendet.
\begin{lstlisting}[language=TeX, caption={Verwendung lstinputlisting}, label={lst:use_input}]
  \lstinputlisting[language=C]{example.c}
\end{lstlisting}
\lstinputlisting[language=C, caption={Beispiel für lstinputlisting},label={lst:input}]{example.c}

\section{Verwendung der \texttt{lstlisting}-Umgebung}
%========
Ein Listing kann auch als Umgebung verwendet werden.
\begin{lstlisting}[language=TeX, caption={Verwendung lstlisting Enviroment}, label={lst:use_enviroment}]
  \begin{lstlisting}[language=Ruby]
    class Hello
      def say_hello
        puts("Hello Hello")
      end
    end
  \end{lstlsting}
\end{lstlisting}
Die Ausgabe sieht wie folgt aus.
\begin{lstlisting}[language=Ruby, caption={Beispiel lstlisting Enviroment}, label={lst:enviroment}]
  class Hello
    def say_hello
      puts("Hello Hello")
    end
  end
\end{lstlisting}

\section{Interessante Links}
%=====
\begin{itemize}
  \item \url{http://texdoc.net/texmf-dist/doc/latex/listings/listings.pdf}
\end{itemize}
\end{document}
