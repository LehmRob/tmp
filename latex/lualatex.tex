%% Dieser Sourcecode ist in der Kodierung UTF-8 zu speichern und mit LuaLaTeX
%% zu kompilieren.
 
\documentclass[a4paper]{scrartcl}
 
\usepackage{polyglossia}
\setdefaultlanguage[spelling=new, babelshorthands=true]{german}
 
\usepackage{fontspec}
\usepackage{unicode-math}
\usepackage{luacode}
\usepackage{hyperref}
 
\setsansfont{Liberation Serif}
\setmonofont{Inconsolata}
%\setmathfont{Cambria Math}
 
\title{Ein Testdokument}
\author{Otto Normalverbraucher}
\date{15. Januar 2014}
 
\begin{document}
 
\maketitle
\tableofcontents
 
\section{Schriftarten}
Mit Lua\TeX{} lassen sich in Windows die systemeigenen Schriftarten verwenden,
wie zum Beispiel Cambria, die über einen großen Satz mathematischer Zeichen für
die Formeldarstellung verfügt, oder auch \textsf{Calibri} und \texttt{Consolas}.
 
\section{Formeln}
Lua\TeX{} ist auch ohne Formeln sehr nützlich und einfach zu verwenden. Grafiken,
Tabellen, Querverweise aller Art, Literatur- und Stichwortverzeichnis sind
kein Problem.
 
Formeln sind etwas schwieriger, dennoch hier ein einfaches Beispiel:
 
\begin{displaymath}
  E = \frac{m_{0} c^{2}}{\sqrt{1-v^{2}/c^{2}}}
\end{displaymath}
 
\section{Lua-Code}
Lua\TeX{} kann aber auch Lua-Programmcode ausführen. So erzeugt man zum Beispiel
mit \texttt{directlua} die Zufallszahl \directlua{tex.print(math.random())}.
Auch die Kreiszahl $\pi$ muss man nicht mehr auswendig wissen, sie hat den Wert
\directlua{tex.print(math.pi)}.
 
Mithilfe der \texttt{luacode}-Umgebung kann man sogar zählen, wie hier bis Sechzig:
\begin{luacode}
  for x=1,60 do
    tex.print(x)
  end
\end{luacode}
 
Hier ist der Schluss des Testdokuments.
 
\end{document}