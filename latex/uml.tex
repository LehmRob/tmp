\documentclass[12pt, a4paper]{scrartcl}

\usepackage[utf8]{inputenc}
\usepackage[ngerman]{babel}
\usepackage[T1]{fontenc}
\usepackage{graphicx}
\usepackage{tikz}
\usepackage{tikz-uml}
\usepackage{hyperref}

\title{Beispiel für tikz-uml}
\author{Robert Lehmann}

\begin{document}
\maketitle
\tableofcontents
\listoffigures

\section{Erstellung einer UML-Klasse}
%====
\begin{figure}[htb]
  \centering
  \begin{tikzpicture}
    \umlclass{Example}
    {
      -- String: name
    }
    {
      + getName()
    }
  \end{tikzpicture}
\caption{Erstellung einer UML-Klasse}
\end{figure}

\section{Interessante Links}
%====
\begin{itemize}
  \item \url{http://perso.ensta-paristech.fr/~kielbasi/tikzuml/index.php?lang=en}
\end{itemize}
\end{document}
